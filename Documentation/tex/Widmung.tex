\documentclass[Bachelorarbeit.tex]{subfiles}
\begin{document}
%\chapter*{Widmung}
\chapter*{To Do}
%
\begin{enumerate}
	\item Sequenz Diagramm erstellen
	%\item Projektbeschreibung zu genau \pfeil dies in den Anforderungen beschreiben!
	\item Wie beschreiben wir die Überprüfung der \acrshort{nfr} und \acrshort{fr} Das wichtigste ist hier, denke ich, dass die Regelgüte irgendwo beschrieben wird (Überprüfung).
	\item Use-Cases Beschreiben
	\item Klassendiagramme müssen alle Methoden enthalten
	\item Attribute müssen nicht unbedingt definiert werden
	\item Mit Zeit kann auch gearbeitet werden bei Zustandsautomaten, siehe Beispiel.
\end{enumerate}
%
\section*{Testerei}
%
Test Referenz auf Anforderung mit fixem Label: \ref{req:1}, \ref{req:20}\\
%
\end{document}